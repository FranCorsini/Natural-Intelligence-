\chapter{Experiment}
In this experiment we assessed the effectiveness of the vision for tele-navigation by using radio controlled car. The focus of this experiment lies in the performance of the driver using different distortion of vision. 

\section{Experimental Setup}
To be able to do the experiment, several hardware are needed, namely:

\begin{itemize}
  \item Radio controlled car, including the controller
  \item Wireless camera, such as GoPro
  \item Screen
\end{itemize}

Aside from the hardware, a location such as a maze or a track is also necessary (TODO specify the maze/track). 

(Also: a way to gather the information of the radio controlled car itself?, such as listening to the radio frequency to see the behaviour of the car (how it is being navigated/controlled)

\iffalse
- miniature radio controlled car
- environment with high(relative to car) location-cues
- Batteries
- Car controller
- Way to intercept the driving signals
- GoPro?
- wormeye view
- track / maze

\section{Remote navigation}
- screen
- head mounted display?
- FOV closure/distortion stuff
- remove center / peripheral, letterboxes
\fi

\section{Method}
X amount of participants has to navigate through the maze in x number different combinations. The combinations can be found in Table \ref{measurementcombinations}. The tester has to navigate the car in an another room using a screen which shows the camera feed placed on the top of the car. First the testers are given the time to play around with the radio controlled car to get a feeling on how the car works, after this the tester are asked to perform the task. 

\begin{table}[h]
\begin{tabular}{|l|lll|}
\hline
No & Vision  & Distortion & Screen distance \\\hline
1  & Normal  & None              & ?               \\
2  & Reduced & Scaling           & ?               \\
3  & Reduced & Letterbox           & ?               \\
4  & ...     & ...               & ...             \\\hline
\end{tabular}
\caption{Test combinations}
\label{measurementcombinations}
\end{table}


\section{Measurements}
To measure the performance of the driver the time to finish, the number of collision and the accuracy? are noted for each combination. The result for each combination will be compared and analysed.
\iffalse
- time
- accuracy
- collision
- driver comfort level (This can be achieved by using questionnaires)
- Screen distance
- Screen resolution
- Screen Size
Reduced vs no-reduced.
\fi

\section{Questionnaires}
Aside for the performance measurement mentioned above, we are also interested in the sense of presence of the tester during the experiment. To gather the information for this subject presence questionnaires is used. The IPQ (igroup presence questionnaires) is presented below in Dutch and English. 

NOTE: not quite sure if we should use this one, because iqp is a scale for measuring the sense of presence experienced in a virtual environment and our experiment will not be using virtual environment?).

\begin{table}[h]
\begin{tabular}{ll}
   & Dutch question                                                                                                                                                                     \\
1  & Ik had het gevoel aanwezig te zijn in de computerwereld                                                                                                                            \\
2  & Ik had het gevoel omgeven te zijn door de virtuele wereld                                                                                                                          \\
3  & Ik had het gevoel slechts plaatjes te aanschouwen                                                                                                                                  \\
4  & Ik had niet het gevoel in de virtuele ruimte aanwezig te zijn                                                                                                                      \\
5  & \begin{tabular}[c]{@{}l@{}}Ik had meer het gevoel bezig te zijn in de virtuele ruimte, \\ dan dat ik het gevoel had iets van buitenaf te bedienen\end{tabular}                     \\
6  & Ik voelde me aanwezig in de virtuele ruimte                                                                                                                                        \\
7  & \begin{tabular}[c]{@{}l@{}}Hoe bewust was u zich van de echte omgeving \\ (bv. geluiden van buiten, kamertemperatuur), \\ terwijl u zich bevond in de virtuele ruimte\end{tabular} \\
8  & Ik was me niet bewust van mijn echte omgeving                                                                                                                                      \\
9  & Ik lette nog op de echte omgeving                                                                                                                                                  \\
10 & Ik ging volledig op in de virtuele wereld                                                                                                                                          \\
11 & Hoe echt kwam de virtuele omgeving op u over                                                                                                                                       \\
12 & \begin{tabular}[c]{@{}l@{}}In hoeverre kwam uw ervaring in de virtuele omgeving \\ overeen met uw ervaringen in de echte wereld?\end{tabular}                                      \\
13 & Hoe werkelijk kwam de virtuele wereld op u over                                                                                                                                    \\
14 & \begin{tabular}[c]{@{}l@{}}De virtuele wereld kwam echter op mij over dan de werkelijke\\  wereld\end{tabular}                                                                    
\end{tabular}
\caption{Dutch IPQ}
\end{table}

\begin{table}[h]
\begin{tabular}{lll}
Number & shortcut                                                & English question                                                                                                                                                                           \\
1      & sense of being there                                    & In the computer generated world I had a sense of "being there"                                                                                                                             \\
2      & sense of VE behind                                      & Somehow I felt that the virtual world surrounded me.                                                                                                                                       \\
3      & only pictures                                           & I felt like I was just perceiving pictures.                                                                                                                                                \\
4      & not sense of being in v. space                          & I did not feel present in the virtual space.                                                                                                                                               \\
5      & sense of acting in VE                                   & \begin{tabular}[c]{@{}l@{}}I had a sense of acting in the virtual space, \\ rather than operating something from outside.\end{tabular}                                                     \\
6      & sense of being present in VE                            & I felt present in the virtual space.                                                                                                                                                       \\
7      & awareness of real env.                                  & \begin{tabular}[c]{@{}l@{}}How aware were you of the real world surrounding \\ while navigating in the virtual world? \\ (i.e. sounds, room temperature, other people, etc.)?\end{tabular} \\
8      & not aware of real env.                                  & I was not aware of my real environment.                                                                                                                                                    \\
9      & no attention to real env.                               & I still paid attention to the real environment.                                                                                                                                            \\
10     & attention captivated by VE                              & I was completely captivated by the virtual world.                                                                                                                                          \\
11     & VE real (real/not real)                                 & How real did the virtual world seem to you?                                                                                                                                                \\
12     & experience similar to real env.                         & \begin{tabular}[c]{@{}l@{}}How much did your experience in the virtual \\ environment seem consistent with your\\   real world experience ?\end{tabular}                                   \\
13     & VE real (imagined/real)                                 & How real did the virtual world seem to you?                                                                                                                                                \\
14     & \begin{tabular}[c]{@{}l@{}}VE\\   wirklich\end{tabular} & The virtual world seemed more realistic than the real world.                                                                                                                              
\end{tabular}
\caption{English IPQ}
\end{table}
