\documentclass[a4,11pt]{scrartcl}

\title{Natural Intelligence}
\subtitle{Final Project}
\author{
\begin{tabular}{ll}
\texttt{4409159}&Francesco Corsini\\
\texttt{text}&Dirk Meijer\\
\texttt{text}&Jan-Willem van Velzen\\
\texttt{text}&Edward Teng\\
\end{tabular}} 

\usepackage{float}
\usepackage[margin=1.5in]{geometry}
\usepackage{hyperref}
\usepackage{graphicx}
\usepackage{amsmath}
\usepackage{xcolor}
\usepackage[sfdefault]{cabin}
\usepackage{listings}
\usepackage{caption}
\usepackage{subcaption}
\usepackage{rotating}

\setlength{\parindent}{0pt}
\DeclareTextFontCommand{\emph}{\bf}
\lstset{language=Matlab,
        breaklines=true,
        basicstyle=\small\ttfamily,
        keywordstyle=\color{blue},
        frame=single}


\begin{document}
\maketitle

\null\vfill


\section{Introduction}
During the course of Natural Intelligence our group has been challenged to find a scientific suitable experiment in the vision field to carry out. 
\\ \\
Our attention has immediately focused on remote controlled devices: how do the driver vision effect the driving performance? \\
 Although we initially wanted to experiment on flying drones, and the relation of human controlled navigation to the height, the broader availability of tele-controlled cars made us decide to shift our attention to so called "worm vision" and the relationship between the field of vision and the proficiency at which cars are controlled and moved.
 \\ \\
 Therefore we have decided to set up an experiment to measure how the different changes on the field of vision of the remote driver affect the overall performance. \\ \\
 Performance is a broad term, we therefore analyzed it and provided a specific measure at which it can be calculated. This can be seen in section TODO
\end{document}


